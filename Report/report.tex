%----------------------------------------------------------------------------------------
%	PACKAGES AND DOCUMENT CONFIGURATIONS
%----------------------------------------------------------------------------------------

\documentclass{article}

\usepackage[italian, english]{babel}	% lingua principale inglese con alcune parti in italiano
\usepackage[version=3]{mhchem} % Package for chemical equation typesetting
\usepackage{siunitx} % Provides the \SI{}{} and \si{} command for typesetting SI units
\usepackage{graphicx} % Required for the inclusion of images
\usepackage{natbib} % Required to change bibliography style to APA
\usepackage{amsmath} % Required for some math elements 
\usepackage[utf8]{inputenc}
\usepackage[italian, english]{babel}	% lingua principale inglese con alcune parti in italiano
\usepackage{newlfont}					% codifica il font
\usepackage{color}
\textwidth=450pt\oddsidemargin=0pt

\usepackage{booktabs}					% include tabelle
\usepackage{caption}
\usepackage{graphicx}					% include le figure
\usepackage{float}
\usepackage{subfigure}						% figure multiple

\renewcommand{\varepsilon}{\epsilon}    % convenzione sulle lettere
\renewcommand{\vartheta}{\thetha}		% greche da usare
\renewcommand{\varrho}{\rho}
\renewcommand{\varphi}{\phi}

\usepackage{amsmath}					% pacchetti matematici
\usepackage{amssymb}
\usepackage{amsfonts}
\usepackage{mathptmx}

\renewcommand{\thefootnote}{\fnsymbol{footnote}} % poche note a pie' pagina

\setlength\parindent{0pt} % Removes all indentation from paragraphs

\renewcommand{\labelenumi}{\alph{enumi}.} % Make numbering in the enumerate environment by letter rather than number (e.g. section 6)

%\usepackage{times} % Uncomment to use the Times New Roman font

%----------------------------------------------------------------------------------------
%	DOCUMENT INFORMATION
%----------------------------------------------------------------------------------------

\title{\textbf{An FPGA Implementation of a\\ Deep Variational Autoencoder using hls4ml package}} % Title

\author{Lorenzo \textsc{Valente}} % Author name

\date{\today} % Date for the report

\begin{document}

\maketitle % Insert the title, author and date

\begin{center}
\begin{tabular}{l r}
%Date Performed: & December -, 2021 \\ % Date the experiment was performed
%Partners: & James Smith \\ % Partner names
%& Mary Smith \\
%Instructor: & Professor Smith % Instructor/supervisor
\end{tabular}
\end{center}

% If you wish to include an abstract, uncomment the lines below
% \begin{abstract}
% Abstract text
% \end{abstract}


\section*{Abstract}


%----------------------------------------------------------------------------------------
%	SECTION 1
%----------------------------------------------------------------------------------------

\section{Introduction}

The hls4ml package was developed by members of High Energy Physics (HEP) community
to translate machine learning (ML) algorithms into high level synthesis (HLS) code.
In the project development, hls4ml was used as the tool to perform 
this transformation on a trained \textit{Variational Autoencoder} model.
A schematic workflow is shown in the figure \ref{fig:overview} belown.
 
\begin{figure}[H]
\centering
\includegraphics[scale=0.2]{images/section1/overview.jpg}
\caption{A typical workflow to translate a model into an FPGA implementationusing hls4ml.}
\label{fig:overview}
\end{figure}

The goal of the hls4ml package is to empower a HEP physicist to accelerate
ML algorithms using FPGAs, thanks to its tools for ML models conversion into HLS. 
Indeed, hls4ml makes the translation of Python objects into HLS, and its
synthesis automatic workflow, allowing fast deployment times also for
those who know how to write software or are not yet experts on FPGAs.


%----------------------------------------------------------------------------------------
%	SECTION 2
%----------------------------------------------------------------------------------------

\section{Model Implementation}



%----------------------------------------------------------------------------------------
%	SECTION 3
%----------------------------------------------------------------------------------------

\section{Training and compression}



%----------------------------------------------------------------------------------------
%	SECTION 4
%----------------------------------------------------------------------------------------

\section{FPGA Implementation of the Model}
\subsection{Laboratory Implementation}



%----------------------------------------------------------------------------------------
%	SECTION 4
%----------------------------------------------------------------------------------------

\section{Experimental Results and discussion}




%----------------------------------------------------------------------------------------
%	CONCLUSIONS
%----------------------------------------------------------------------------------------

\section{Conclusions}






%----------------------------------------------------------------------------------------
%	BIBLIOGRAPHY
%----------------------------------------------------------------------------------------



\end{document}